\chapter{Wstęp i cel Pracy}

\section{Wstęp}
\noindent Współczesne technologie znacząco zmieniły sposób, w jaki tworzone i oceniane są teksty, w tym prace naukowe. Wraz z dynamicznym rozwojem narzędzi wspierających redakcję oraz poprawność tekstów, pojawiły się systemy automatycznej korekty pisma, znane jako AWCF (Automated Writing Corrective Feedback). Są to narzędzia, które automatycznie identyfikują i sugerują poprawki błędów w tekstach, obejmując zarówno aspekty gramatyczne, jak i stylistyczne. AWCF mają na celu nie tylko poprawę jakości języka, ale także wspomaganie autorów w rozwijaniu ich umiejętności pisarskich poprzez natychmiastową informację zwrotną.

Systemy te zyskały popularność dzięki swojej wszechstronności i dostępności. Oferują one możliwość szybkiego wychwycenia błędów, oszczędzając czas zarówno autorom, jak i recenzentom. Mimo to, wyzwania związane z automatyczną analizą tekstów wciąż są obecne. Problemy pojawiają się w przypadku złożonych struktur językowych, subtelnych różnic kontekstowych oraz zróżnicowanych stylów pisania. Automatyczne systemy korekty nie zawsze są w stanie w pełni uwzględnić intencje autora, co może prowadzić do błędnych sugestii. Dlatego w dalszym ciągu istnieje potrzeba udoskonalania tych narzędzi, aby mogły one lepiej rozpoznawać kontekst i semantykę tekstu.

\section{Cel Pracy}
Celem niniejszej pracy inżynierskiej jest opracowanie i implementacja automatycznego systemu do sprawdzania prac inżynierskich. Pozwoli on na efektywne weryfikowanie dokumentów pod kątem zgodności z wytycznymi redakcyjnymi oraz poprawności językowej. System ten, bazując na technologiach takich jak Python, PyQT5, zeroGPT i innych będzie w stanie analizować wczytane dokumenty w formacie PDF, oferując automatyczne poprawki oraz wskazówki dotyczące aspektów gramatycznych, jak i strukturalnych pracy. Jego główną funkcjonalnością będzie ocena zgodności prac z przyjętymi standardami akademickimi. To rozwiązanie wspomoże studentów podczas nauki pisania pracy dyplomowej, oraz da narzędzie wykładowcą do zautomatyzowanego sprawdzania prac dyplomowych.

\section{Układ Pracy}
W rozdziale pierwszym została omówiona ogólna tematyka związana z systemami automatycznej korekty pisowni wraz z celem i układem pracy dyplomowej. Rozdział drugi zawiera przegląd najnowszych technologii. Rozdział trzeci obejmuje strukturę programu oraz wymienia wszystkie komponenty z którego jest złożony. TO BE CONTINUE....