\documentclass{pginz}

%%%%%%%%%Miejsce na dodatkowe pakiety%%%%%%%%%%%%%
\usepackage{subcaption}
\usepackage{ragged2e}
\usepackage[T1]{fontenc}
\usepackage[utf8]{inputenc}
\usepackage[polish]{babel}

\begin{document}


\includepdf[page=1]{StronaTytułowa.pdf}
%\includepdf[page={1}]{Oswiadczenie.pdf}


\setcounter{page}{3}

\chapter*{Streszczenie}


\justifying

Współczesne technologie znacząco ułatwiają proces tworzenia i oceny dokumentów, w tym prac naukowych. Wraz z rozwojem systemów do automatycznej oceny tekstów, pojawiły się narzędzia wspomagające zarówno proces redakcji, jak i sprawdzania zgodności z wymaganiami edytorskimi. Jednym z takich rozwiązań są systemy AWCF (Automated Writing Corrective Feedback), które oferują automatyczne poprawianie błędów oraz wskazywanie niezgodności w treści pod kątem poprawności językowej i formalnej. Przykładem takiego systemu jest Grammarly, który zyskał popularność dzięki możliwości oceny i poprawy tekstów w języku angielskim, zapewniając automatyczne poprawki dotyczące gramatyki, interpunkcji oraz stylu pisania.

Niniejsza praca inżynierska dotyczy opracowania systemu do automatycznego sprawdzania prac inżynierskich, który umożliwia wczytywanie dokumentów w formacie PDF i analizowanie ich pod kątem zgodności z kryteriami redakcyjnymi oraz wytycznymi dotyczącymi struktury i poprawności językowej. Aplikacja okienkowa, zbudowana przy użyciu języka Python oraz bibliotek PyQT5 i LanguageTool, pozwala na automatyczne przetwarzanie wczytanych dokumentów. System, bazując na wczytanym pliku, przeprowadza analizę tekstu, sprawdzając między innymi poprawność gramatyczną, zgodność ze strukturą akademicką, oraz inne istotne aspekty, które są wymagane przy ocenie prac dyplomowych.

Główne zalety proponowanego systemu to automatyzacja procesu sprawdzania oraz możliwość szybkiej weryfikacji pracy przez użytkownika. Implementacja systemu opiera się na mechanizmach AWCF, które stanowią kluczowy element oceny poprawności treści. System ten nie tylko wspomaga korektę, ale również przyczynia się do zwiększenia świadomości użytkownika na temat jego błędów i obszarów wymagających poprawy. Inspiracją dla opracowania systemu była potrzeba stworzenia narzędzia, które ułatwi studentom samodzielne przygotowywanie prac dyplomowych zgodnych z wymaganiami edytorskimi oraz promotorom sprawniejsze ich ocenianie.

Celem pracy jest zaprojektowanie i implementacja aplikacji, która będzie mogła być wykorzystywana przez studentów oraz nauczycieli akademickich, oferując szybką i efektywną ocenę jakości prac inżynierskich. W pracy zostaną omówione techniczne aspekty związane z realizacją projektu, w tym zastosowane technologie, algorytmy oraz analiza skuteczności proponowanego rozwiązania.

\bigskip

\noindent\textbf{Słowa kluczowe:} automatyczna ocena tekstu, AWCF, Python, PyQT5, LanguageTool, analiza prac inżynierskich, ocena gramatyczna, systemy wspomagania pisania, analiza PDF, poprawność językowa.

\bigskip

\noindent\textbf{Dziedzina nauki i techniki zgodna z OECD} Engineering and technical sciences, Computer science, Educational technology, Computational linguistics
\chapter*{Abstract}


\justifying

Modern technologies significantly facilitate the process of creating and evaluating documents, including scientific papers. With the development of systems for automatic text evaluation, tools have emerged to support both the editing process and checking for compliance with editorial requirements. One such solution is AWCF (Automated Writing Corrective Feedback) systems, which offer automatic correction of errors and indication of inconsistencies in content for linguistic and formal correctness. An example of such a system is Grammarly, which has gained popularity for its ability to evaluate and correct texts in English, providing automatic corrections for grammar, punctuation and writing style.

This engineering thesis is concerned with the development of a system for the automatic checking of engineering papers that can load documents in PDF format and analyze them for compliance with editorial criteria and guidelines for structure and linguistic correctness. The windowed application, built using the Python language and the PyQT5 and LanguageTool libraries, allows automatic processing of loaded documents. Based on the loaded file, the system performs text analysis, checking, among other things, grammatical correctness, compliance with the academic structure, and other relevant aspects that are required in the evaluation of theses.

The main advantages of the proposed system are the automation of the checking process and the user's ability to quickly verify the work. The implementation of the system is based on AWCF mechanisms, which is a key element of content correctness assessment. The system not only supports proofreading, but also contributes to the user's awareness of his errors and areas for improvement. The development of the system was inspired by the need to create a tool that will help students prepare their theses independently and in accordance with editorial requirements, and promoters to evaluate them more efficiently.

The aim of the work is to design and implement an application that can be used by students and academics, offering quick and efficient assessment of the quality of engineering theses. The paper will discuss the technical aspects related to the implementation of the project, including the technologies used, algorithms and analysis of the effectiveness of the proposed solution.

Translated with DeepL.com (free version)

\bigskip

\noindent\textbf{Keywords:} automatic text evaluation, AWCF, Python, PyQT5, LanguageTool, engineering paper analysis, grammar evaluation, writing support systems, PDF analysis, language correctness.

\bigskip

\noindent\textbf{OECD consistent field of science and technology classification:} Nauki inżynieryjne i techniczne, Elektrotechnika, elektronika i inżynieria informatyczna, Robotyka i Automatyka


\tableofcontents
\addcontentsline{toc}{chapter}{Spis treści}

\include{Symbole}
\chapter*{Lista skrótów}

\begin{itemize}[noitemsep,topsep=0pt,parsep=0pt,partopsep=0pt,labelwidth=1cm,align=left,itemindent=0pt]
\item[AWCF] - (ang. \textit{Automated written corective feedback}) Automatyczna pisemna informacja zwrotna
\item[AWC] - (ang. \textit{Automative written corective}) Automatyczna pisemna korekta
\end{itemize}

\chapter{Wstęp i cel Pracy}

\section{Wstęp}
\noindent Współczesne technologie znacząco zmieniły sposób, w jaki tworzone i oceniane są teksty, w tym prace naukowe. Wraz z dynamicznym rozwojem narzędzi wspierających redakcję oraz poprawność tekstów, pojawiły się systemy automatycznej korekty pisma, znane jako AWCF (Automated Writing Corrective Feedback). Są to narzędzia, które automatycznie identyfikują i sugerują poprawki błędów w tekstach, obejmując zarówno aspekty gramatyczne, jak i stylistyczne. AWCF mają na celu nie tylko poprawę jakości języka, ale także wspomaganie autorów w rozwijaniu ich umiejętności pisarskich poprzez natychmiastową informację zwrotną.

Systemy te zyskały popularność dzięki swojej wszechstronności i dostępności. Oferują one możliwość szybkiego wychwycenia błędów, oszczędzając czas zarówno autorom, jak i recenzentom. Mimo to, wyzwania związane z automatyczną analizą tekstów wciąż są obecne. Problemy pojawiają się w przypadku złożonych struktur językowych, subtelnych różnic kontekstowych oraz zróżnicowanych stylów pisania. Automatyczne systemy korekty nie zawsze są w stanie w pełni uwzględnić intencje autora, co może prowadzić do błędnych sugestii. Dlatego w dalszym ciągu istnieje potrzeba udoskonalania tych narzędzi, aby mogły one lepiej rozpoznawać kontekst i semantykę tekstu.

\section{Cel Pracy}
Celem niniejszej pracy inżynierskiej jest opracowanie i implementacja automatycznego systemu do sprawdzania prac inżynierskich. Pozwoli on na efektywne weryfikowanie dokumentów pod kątem zgodności z wytycznymi redakcyjnymi oraz poprawności językowej. System ten, bazując na technologiach takich jak Python, PyQT5, zeroGPT i innych będzie w stanie analizować wczytane dokumenty w formacie PDF, oferując automatyczne poprawki oraz wskazówki dotyczące aspektów gramatycznych, jak i strukturalnych pracy. Jego główną funkcjonalnością będzie ocena zgodności prac z przyjętymi standardami akademickimi. To rozwiązanie wspomoże studentów podczas nauki pisania pracy dyplomowej, oraz da narzędzie wykładowcą do zautomatyzowanego sprawdzania prac dyplomowych.

\section{Układ Pracy}
W rozdziale pierwszym została omówiona ogólna tematyka związana z systemami automatycznej korekty pisowni wraz z celem i układem pracy dyplomowej. Rozdział drugi zawiera przegląd najnowszych technologii. Rozdział trzeci obejmuje strukturę programu oraz wymienia wszystkie komponenty z którego jest złożony. TO BE CONTINUE....1
\chapter{Przegląd obecnych technologi}
\label{chap:przeglad}

\section{LanguageTool}
LanguageTool to otwartoźródłowe narzędzie do sprawdzania gramatyki, stylu oraz pisowni, obsługujące ponad 25 języków. Działa zarówno jako aplikacja samodzielna, jak i wtyczka do przeglądarek, edytorów tekstu oraz narzędzi online, oferując natychmiastową analizę tekstu pod kątem błędów językowych. Jego funkcjonalność obejmuje sprawdzanie interpunkcji, zgodności stylistycznej oraz rozbudowanych reguł gramatycznych. LanguageTool oferuje darmową wersję z podstawowymi funkcjami oraz płatną wersję premium, która zapewnia bardziej zaawansowaną korektę i dodatkowe funkcje.

\section{PyQT5}
PyQt5 to zestaw narzędzi do tworzenia graficznych interfejsów użytkownika (GUI) w Pythonie, który umożliwia korzystanie z funkcji Qt, popularnej platformy do tworzenia aplikacji o interfejsie graficznym. PyQt5 pozwala na projektowanie aplikacji wieloplatformowych, które działają zarówno na systemach Windows, macOS, jak i Linux. Biblioteka ta zawiera gotowe komponenty, takie jak przyciski, okna dialogowe, pola tekstowe i wiele innych, umożliwiając szybkie tworzenie aplikacji.

Dzięki obsłudze zarówno tradycyjnego programowania, jak i bardziej wizualnych narzędzi (np. Qt Designer), PyQt5 ułatwia budowanie złożonych interfejsów. PyQt5 jest dostępna za darmo w ramach licencji GPL.

\section{GPTzero}
ZeroGPT to zaawansowane narzędzie przeznaczone do identyfikacji treści wygenerowanych przez modele językowe oparte na sztucznej inteligencji, takie jak ChatGPT. Aplikacja działa poprzez analizowanie wskaźników takich jak „perplexity” (zakłopotanie) i „burstiness” (zrywność) — parametrów, które pozwalają rozróżnić, czy dany tekst został stworzony przez człowieka, czy też wygenerowany przez algorytm AI. ZeroGPT jest stosowane głównie w środowiskach akademickich i wydawniczych w celu potwierdzania autentyczności tekstów, co jest kluczowe w walce z plagiatem i w procesie weryfikacji autorstwa.

Dzięki API firmy zeroGPT możliwa jest interakcja ich usług do autorskich rozwiązań. Platforma udostępnia listę endpointów, za pomocą których możliwa jest komunikacja z serwerem. Jedną z wielu funkcjonalności jest przesył tekstu w postaci string do serwera i następująca po tym odpowiedź wskazując wynik charakteryzując w jakim stopniu dany tekst został napisany przez człowieka albo model językowy.

\section{PyMuPDF}
PyMuPDF to biblioteka w języku Python, zaprojektowana do pracy z plikami w formacie PDF oraz innymi formatami dokumentów, takimi jak XPS, EPUB i CBZ. Dzięki funkcjom, takim jak renderowanie, ekstrakcja tekstu, manipulacja stronami oraz dodawanie znaków wodnych, PyMuPDF umożliwia programistom szybkie i precyzyjne przetwarzanie dokumentów. Biblioteka ta jest wykorzystywana w różnorodnych aplikacjach, od automatycznego przetwarzania dokumentów po systemy zarządzania treścią.
\include{Rozdzial3}
% tu będą kolejne rozdziały

\listoffigures
\addcontentsline{toc}{chapter}{Spis rysunków}
\listoftables
\addcontentsline{toc}{chapter}{Spis tabel}


%alternatywa - bibtex
\begin{thebibliography}{20}
\bibitem{koltovskaia} Koltovskaia, S. (2020). Student engagement with automated written corrective feedback (AWCF) provided by Grammarly: A multiple case study. Assesing Writing, 44, 100450.Habibzadeh, F..
\bibitem{GPTzeroPer} (2023). GPTZero performance in identifying artificial intelligence-generated medical texts: a preliminary study. Journal of Korean Medical Science, 38(38).
\end{thebibliography}

%*****************
%wymagane dodatki:
% Opis dyplomu
% Zawartość płyty CD
% Instrukcja dla projektanta
% Instrukcja dla użytkownika
\begin{appendices}
%\include{AppA}
%\include{AppB}
\end{appendices}
%*****************

\end{document}