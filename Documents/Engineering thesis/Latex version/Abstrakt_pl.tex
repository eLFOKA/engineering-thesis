\chapter*{Streszczenie}


\justifying

Współczesne technologie znacząco ułatwiają proces tworzenia i oceny dokumentów, w tym prac naukowych. Wraz z rozwojem systemów do automatycznej oceny tekstów, pojawiły się narzędzia wspomagające zarówno proces redakcji, jak i sprawdzania zgodności z wymaganiami edytorskimi. Jednym z takich rozwiązań są systemy AWCF (Automated Writing Corrective Feedback), które oferują automatyczne poprawianie błędów oraz wskazywanie niezgodności w treści pod kątem poprawności językowej i formalnej. Przykładem takiego systemu jest Grammarly, który zyskał popularność dzięki możliwości oceny i poprawy tekstów w języku angielskim, zapewniając automatyczne poprawki dotyczące gramatyki, interpunkcji oraz stylu pisania.

Niniejsza praca inżynierska dotyczy opracowania systemu do automatycznego sprawdzania prac inżynierskich, który umożliwia wczytywanie dokumentów w formacie PDF i analizowanie ich pod kątem zgodności z kryteriami redakcyjnymi oraz wytycznymi dotyczącymi struktury i poprawności językowej. Aplikacja okienkowa, zbudowana przy użyciu języka Python oraz bibliotek PyQT5 i LanguageTool, pozwala na automatyczne przetwarzanie wczytanych dokumentów. System, bazując na wczytanym pliku, przeprowadza analizę tekstu, sprawdzając między innymi poprawność gramatyczną, zgodność ze strukturą akademicką, oraz inne istotne aspekty, które są wymagane przy ocenie prac dyplomowych.

Główne zalety proponowanego systemu to automatyzacja procesu sprawdzania oraz możliwość szybkiej weryfikacji pracy przez użytkownika. Implementacja systemu opiera się na mechanizmach AWCF, które stanowią kluczowy element oceny poprawności treści. System ten nie tylko wspomaga korektę, ale również przyczynia się do zwiększenia świadomości użytkownika na temat jego błędów i obszarów wymagających poprawy. Inspiracją dla opracowania systemu była potrzeba stworzenia narzędzia, które ułatwi studentom samodzielne przygotowywanie prac dyplomowych zgodnych z wymaganiami edytorskimi oraz promotorom sprawniejsze ich ocenianie.

Celem pracy jest zaprojektowanie i implementacja aplikacji, która będzie mogła być wykorzystywana przez studentów oraz nauczycieli akademickich, oferując szybką i efektywną ocenę jakości prac inżynierskich. W pracy zostaną omówione techniczne aspekty związane z realizacją projektu, w tym zastosowane technologie, algorytmy oraz analiza skuteczności proponowanego rozwiązania.

\bigskip

\noindent\textbf{Słowa kluczowe:} automatyczna ocena tekstu, AWCF, Python, PyQT5, LanguageTool, analiza prac inżynierskich, ocena gramatyczna, systemy wspomagania pisania, analiza PDF, poprawność językowa.

\bigskip

\noindent\textbf{Dziedzina nauki i techniki zgodna z OECD} Engineering and technical sciences, Computer science, Educational technology, Computational linguistics