\chapter{Przegląd obecnych technologi}
\label{chap:przeglad}

\section{LanguageTool}
LanguageTool to otwartoźródłowe narzędzie do sprawdzania gramatyki, stylu oraz pisowni, obsługujące ponad 25 języków. Działa zarówno jako aplikacja samodzielna, jak i wtyczka do przeglądarek, edytorów tekstu oraz narzędzi online, oferując natychmiastową analizę tekstu pod kątem błędów językowych. Jego funkcjonalność obejmuje sprawdzanie interpunkcji, zgodności stylistycznej oraz rozbudowanych reguł gramatycznych. LanguageTool oferuje darmową wersję z podstawowymi funkcjami oraz płatną wersję premium, która zapewnia bardziej zaawansowaną korektę i dodatkowe funkcje.

\section{PyQT5}
PyQt5 to zestaw narzędzi do tworzenia graficznych interfejsów użytkownika (GUI) w Pythonie, który umożliwia korzystanie z funkcji Qt, popularnej platformy do tworzenia aplikacji o interfejsie graficznym. PyQt5 pozwala na projektowanie aplikacji wieloplatformowych, które działają zarówno na systemach Windows, macOS, jak i Linux. Biblioteka ta zawiera gotowe komponenty, takie jak przyciski, okna dialogowe, pola tekstowe i wiele innych, umożliwiając szybkie tworzenie aplikacji.

Dzięki obsłudze zarówno tradycyjnego programowania, jak i bardziej wizualnych narzędzi (np. Qt Designer), PyQt5 ułatwia budowanie złożonych interfejsów. PyQt5 jest dostępna za darmo w ramach licencji GPL.

\section{GPTzero}
ZeroGPT to zaawansowane narzędzie przeznaczone do identyfikacji treści wygenerowanych przez modele językowe oparte na sztucznej inteligencji, takie jak ChatGPT. Aplikacja działa poprzez analizowanie wskaźników takich jak „perplexity” (zakłopotanie) i „burstiness” (zrywność) — parametrów, które pozwalają rozróżnić, czy dany tekst został stworzony przez człowieka, czy też wygenerowany przez algorytm AI. ZeroGPT jest stosowane głównie w środowiskach akademickich i wydawniczych w celu potwierdzania autentyczności tekstów, co jest kluczowe w walce z plagiatem i w procesie weryfikacji autorstwa.

Dzięki API firmy zeroGPT możliwa jest interakcja ich usług do autorskich rozwiązań. Platforma udostępnia listę endpointów, za pomocą których możliwa jest komunikacja z serwerem. Jedną z wielu funkcjonalności jest przesył tekstu w postaci string do serwera i następująca po tym odpowiedź wskazując wynik charakteryzując w jakim stopniu dany tekst został napisany przez człowieka albo model językowy.

\section{PyMuPDF}
PyMuPDF to biblioteka w języku Python, zaprojektowana do pracy z plikami w formacie PDF oraz innymi formatami dokumentów, takimi jak XPS, EPUB i CBZ. Dzięki funkcjom, takim jak renderowanie, ekstrakcja tekstu, manipulacja stronami oraz dodawanie znaków wodnych, PyMuPDF umożliwia programistom szybkie i precyzyjne przetwarzanie dokumentów. Biblioteka ta jest wykorzystywana w różnorodnych aplikacjach, od automatycznego przetwarzania dokumentów po systemy zarządzania treścią.